\section{Quantum GIS 1.0}
\pagenumbering{arabic}
\setcounter{page}{1}

Quantum GIS (QGIS) is a user friendly Geographic Information System (GIS).
It is written in C++ and Python with a QT based GUI. It is licensed under the
GNU General Public License (GPL) and an official project of the Open Source
Geospatial Foundation (OSGeo). The current stable version 1.0 was released in
January 2009. 

\minisec{History}

The QGIS project started in February of 2002, with the first release in June
of the same year. The initial goal was to create a viewer for PostGIS data
that ran on GNU/Linux. From those beginnings, QGIS has become a true
cross-platform application that runs on all major versions of Unix,
GNU/Linux, as well as Mac OS and MS Windows. It supports numerous vector,
raster, and database formats and provides a wide variety of core and external
geoprocessing functionalities.

\subsection{QGIS Open Source Community}

The QGIS project is carried out largely by a group of dedicated developers,
translators, documenters, release helpers, bug reporters, and promoters. It
is managed by the Project Steering Committee (PSC), a five member committee
providing technical guidance, community liason, release management, and
financial/marketing activities. The work of the QGIS project process is
spread between numerous people who each have a specific area of
resposibility. 

All these volunteers together with a large number of users make up the
world-wide QGIS comunity, which in time has built up a comprehensive,
valuable and useful code and documentation base free to use and improve for
everybody.

\begin{figure}[h]
   \begin{center}
   \caption{QGIS Community Map}\label{fig:community-map}\smallskip
   \includegraphics[clip=true, width=10cm]{community-map}
\end{center}
\end{figure}

With community platforms such as website, wiki, and blog the QGIS project
provides latest news, release, usage, and development information and also
allows own contributions after subscription. A prefered first step to contact
other users for discussions of QGIS in general, as well as specific questions
regarding its installation and use is by joining the qgis-user mailing list,
the QGIS Forum or the Internet Relay Chat (IRC).

\subsection{Functionality}

QGIS offers a growing number of common GIS functionalities provided by core
features and plugins. They are presented in a friendly graphical user
interface, clearly seperated in menu bar, tool bar, status bar, map view,
overview, and map legend (see Figure~\ref{fig:qgis10}).

\begin{figure}[h]
   \begin{center}
   \caption{Quantum GIS 1.0.0 'Kore'}\label{fig:qgis10}\smallskip
   \includegraphics[clip=true, width=\textwidth]{qgis10}
\end{center}
\end{figure}

The menu and tool bar give access to all QGIS features and at a glance
provide following features: 

\begin{itemize}
\item view and overlay vector and raster layer in different formats and
projections without conversion to an internal or common format. Supported are
PostgreSQL/PostGIS, GDAL/OGR supported vector and raster layers such as ESRI
Shapefile, MapInfo, GML, GeoTiff or Erdas Img., GRASS locations, and
OGC-compliant WMS and WFS;
\item interactively explore data, including features such as on the fly
(OTF) projection, identify/select geometries, view, select and search
attributes, label features, change vector and raster symbology; 
\item compose print layouts adding map canvas, legend, scalebar, images and
text lables in a print composer plugin;
\item create, edit, manage and export vector layers into several formats.
Raster layer have to be imported into GRASS GIS to be edited and
exported. 
\item perform spatial geoprocessing on PostgreSQL/PostGIS and other OGR
supported vector layers including overlay, buffer, sampling, geometry and
database management. The integrated GRASS Plugin allows to include the
complete GRASS functionality of more than 300 modules.
\end{itemize}

\subsection{Plugin Architecture}

QGIS has been designed with a plugin architecture and therefore new
customized features and functions can easily be added to the application.
Many of the features in QGIS are actually implemented as core or external
plugins. 

\textbf{Core Plugins} are maintained by the QGIS Development Team. They are
written in C++ or Python, are automatically part of every QGIS distribution
and can be loaded with the Plugin Manager. There are currently 17 core
plugins available, such as GRASS GIS integration (See
Figure~\ref{fig:grass-plugin}), Georeferencer, Mapserver Export, Shapefile to
PostGIS Import Tool, OGR Layer Converter, GPS Tools, Add Delimited Text Layer
or WFS support:  

\begin{figure}[h]
   \begin{center}
   \caption{QGIS Core Plugin (GRASS GIS Integration)}
    \label{fig:grass-plugin}\smallskip
   \includegraphics[clip=true, width=14cm]{grass-plugin}
\end{center}
\end{figure}

\textbf{External Plugins} are all written in Python. They are stored in
an official, external repository and maintained by the individual author. The
user can easily add those Plugins to QGIS with the Python Plugin Installer
(See Figure~\ref{fig:python-plugin}).

\begin{figure}[h]
   \begin{center}
   \caption{QGIS Python Plugin Installer}\label{fig:python-plugin}\smallskip
   \includegraphics[clip=true, width=\textwidth]{python-plugin-installer}
\end{center}
\end{figure}


\subsection{Development}

QGIS 1.0 provides a stable API from which custom solutions in Python or C++
can be developed. Even though 1.0 is pretty fresh, there are already a number
of exciting developments underway in both the core application and plugins.

\subsection{Perspective / Conclusion}

\minisec{Authors}

The authors of this article are QGIS Project Steering Committee Members:

Otto Dassau <dassau@nature-consult.de>  
\\Gary Sherman <sherman@mrcc.com>
\\Tim Sutton <tim@linfinity.com>
\\Marco Hugentobler <marco.hugentobler@karto.baug.ethz.ch>
\\Paolo Cavallini <cavallini@faunalia.it>

\minisec{Links}

For more information, have a look at the following website:

Quantum GIS project: http://qgis.org
\\QGIS Forum: http://forum.qgis.org
\\QGIS Blog: http://blog.qgis.org
\\QGIS User Mailing List: http://lists.osgeo.org/mailman/listinfo/qgis-user
\\QGIS IRC: Channel \#qgis port 6667 at irc.freenode.net
\\Open Source Geospatial Foundation: http://www.osgeo.org
 


 



